\documentclass{article}
\usepackage{amsmath}
\usepackage{graphicx}
\usepackage{amssymb}
\usepackage{mhchem}
\usepackage{chemfig}
\author{Ramakrishna Kompella}

\begin{document}

\section*{Sets and Relations}

A set is a bunch of things. It could be people, animals, or anything.  
You can have sets of all smartphones manufactured, set of all fictional characters, set of all books.

Sets have elements or members. As we're just mentioning they could be things or even other sets.  
A line is a set of points.  
A plane/surface/rectangle is a set of lines.

If a set $A$ contains the element $n$ then we say that $n$ belongs to $A$, or: $n \in A$.

Note: lowercase letters are often used for elements and uppercase for sets.

\section*{Extension}

Now, take say that two sets which have the same elements are the same.  
Although it seems trivial, that need not be so.  
For example, the order of the elements might matter. However, for a set, they do not.

Another more appropriate although less compelling example:

Consider a person as a set of all his ancestors.  
Although each person may have only one set describing them.  
Different people may have the same ancestors i.e.\ siblings.

A more general way to understand this would be to say that there is nothing else to a set than the elements belonging to it.

More formally,

Axiom of extension: A set is determined by its extension.

\section*{Subsets and supersets}

Let's take two sets: $A$ and $B$.  
If every element in $A$ is an element of $B$ (not necessarily the other way around),  
that is, there could be some more elements in $B$ other than the ones in $A$.

We say that $A$ is a subset of $B$, or $B$ includes $A$:

\[
A \subset B
\]

or

\[
B \supset A
\]

We see that, by definition $A \subset A$.  
If $A \subset B$ and $A \neq B$, then $A$ is a proper subset.  
We can also infer the following:

If $A \subseteq B$ and $B \subseteq C$, then $A \subseteq C$.

If $A \subseteq B$ and $B \subseteq A$, then $A = B$.

Instead, the second condition is often used to prove equality.

Belonging and inclusion are different things.

\end{document}